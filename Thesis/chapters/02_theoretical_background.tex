% !TeX root = ../main.tex
% Add the above to each chapter to make compiling the PDF easier in somexcv editors.

\chapter{Theoretical Background}\label{chapter:theoretical_background}

Machine Learning \cite{wang2016machine, mahesh2020machine} is a rapidly evolving field of artificial intelligence. There are different types of algorithms that are used for specific tasks involving supervised learning where the algorithm maps inputs to the given labels, unsupervised learning where the labels to the input are not available, and semi-supervised learning which combines labeled and unlabeled data. Additionally, there is reinforcement learning where the model learns by observing the environment \cite{ayodele2010types}. Specific tasks are e.g. classification where input has to be assigned to specific classes, regression where input has to be assigned to a continuous value (both supervised) and clustering (unsupervised) where the goal is to group the input. 

There are many different algorithms that accomplish these goals, for example support vector machines \cite{noble2006support}, the tsetlin machine \cite{granmo2018tsetlin}, and decision trees \cite{rokach2005decision}. One very important class of algorithms is \textit{artificial neural networks} \ref{sec:neural_networks}. After the introduction to neural networks, the hyperparameter optimization is presented with different techniques to improve machine learning models. In the following, sparse grids are presented which will be needed as foundation to hyperparameter optimization of neural networks with sparse grids. 

\section{Introduction to Neural Networks}\label{sec:neural_networks}

Neural networks \cite{bishop1994neural, da2017artificial} are very  powerful for using various tasks. They are very versatile and they exist in very different variations, ranging from a very small size up to very large networks for more complex tasks.

The smallest part of a neural network is the \textit{perceptron}. A network consisting only of one perceptron can be seen in Figure \ref{fig:perceptron}.

\begin{figure}[htbp!]
	\centering
	\includegraphics[scale=0.18]{figures/perceptron.png}
	\caption{ Neural network consisting of only one perceptron. Taken from \cite{da2017artificial}. }
	\label{fig:perceptron}
\end{figure}

The output $ y $ is computed with 

\begin{equation}
	u = \sum_{i=1}^n w_i \cdot x_i - \theta  \text{,   } y = g(u).
\end{equation}

The network has $ n $ inputs $ x_i $ and weights $ w_i $. $ \theta $ is the activation threshold (also called bias), $ g $ is the activation function, and $ u $ is the activation potential \cite{da2017artificial}. 

\section{Hyperparameter Optimization}

Most machine learning models have parameters that have to be defined before the learning phase. They are called hyperparameters and strongly influence the behavior of the model. One example is the number of epochs of the learning phase of a neural network. There are different techniques for the optimization of hyperparameters and they all define the machine learning model as a black box function $ f $ with the hyperparameters as input and the resulting performance as output. The overall goal is to find a configuration $ \lambda_{min} $ from $ \Lambda = \Lambda_1 \times \Lambda_2 \times ... \times \Lambda_N $ that minimizes the function $ f $ with $ N $ hyperparameters with 
\begin{equation}
	\label{eq:optimization}
	\lambda_{min} = \text{arg} \min_{\lambda \in \Lambda} f(\lambda) .
\end{equation}

In our case, the function f is a machine learning algorithm that is trained on a training set and evaluated on a test set. With this, the minimization of e.g. the loss of the model optimizes the decisions it is making which leads to better prediction results. Note that one function evaluation of f is usually very expensive as the training of a machine learning model with many parameters and weights takes much time. The data set consists of $ \{ (x_i, y_i) | x_i \in X, y_i \in Y, 0 \le i \le m \} $ with m being the number of data samples. The $ x_i $ is the input data to the model and the goal is that 
\begin{equation}
	\forall i: M(x_i) = y_i.
\end{equation}
where M is the model. In the context of supervised learning, the whole data set is split into a training set which is used to optimize the model and a testing set to evaluate the performance on new, unseen data \cite{supervised_learning}.

All in all, the goal is get evaluation scores on the testing data set which can be achieved with Equation \ref{eq:optimization}. 
\cite{feurer2019hyperparameter,bischl2021hyperparameter,yang2020hyperparameter}

In the following, different techniques for the optimization are presented and discussed with their advantages and disadvantages.

\subsection{Grid Search}
The idea of the first approach for the optimization is to discretize the domains of each hyperparameter and evaluate each combination. This suffers from the curse of the dimensionality as it scales exponentially with the number of hyperparameters. For $ d $ parameters and $ n $ values per hyperparameter, $ n^d $ different configurations are possible which all have to be evaluated. 

One advantage of this method is that it is easy to implement and very simple. Also, the whole search space is explored evenly.

On the other hand, the curse of the dimensionality makes it very slow if the function evaluations are very expensive which is the case for most machine learning algorithms. Another drawback is that each hyperparameter only takes $ n $ different values. The comparison to random search can be seen in Figure \ref{fig:comparison_searches}.


\subsection{Random Search}
The next technique \cite{random_search} is similar to the grid search because the idea is also to evaluate different hyperparameter configurations. In contrast to the previous one, random search generates for each run and for each parameter exactly one random value from an interval which has to be specified. For this approach, a budget $ b $ has to be given. This parameter determines the number of different combinations that are evaluated. A direct comparison of grid search and random search can be seen in figure \ref{fig:comparison_searches}.

\begin{figure}[hbtp!]
	\centering
	\includegraphics[scale=0.2]{figures/comparison_searches.png}
	\caption{Comparison of grid search (left) and random search (right) in the two dimensional case. For both techniques, 9 different combinations are evaluated. In the left case, only 3 distinct values for each hyperparameter are set whereas there are 9 different values for each parameter in the random search. Taken from \cite{feurer2019hyperparameter}. }
	\label{fig:comparison_searches}
\end{figure}

In this figure, a two dimensional setting is depicted. For both techniques, 9 different combinations are evaluated. In the case of grid search, only 3 distinct values are taken for each hyperparameter while there are 9 different ones in the random search. In this example, the better result is found in random search as more distinct values are taken for the important parameter. Note that this is not always the case. 

Compared to the normal grid search, this is one advantage. For each hyperparameter, $ b $ (budget) different values are taken into consideration which is much more compared to the grid search with the same overall number of combinations. Additionally, this technique is also easy to implement and relatively simple. 

One disadvantage is that it is also very expensive if the budget is high because of the long training times of machine learning models. 


\subsection{Bayesian Optimization}

Another possible technique for finding the best hyperparameters of machine learning models is called bayesian optimization (BO) \cite{snoek2012practical}. This is an iterative approach for optimizing the expensive black box function by modeling it based on observations. A so-called \textit{surrogate model} $ \hat{f} $ is made with the help of the \textit{archive} $ A $ which contains observed function evaluations. This surrogate model is created by regression and the technique which is most often used is the Gaussian process \cite{bischl2021hyperparameter} which is only suitable if the number of hyperparameters is not too high \cite{andonie2019hyperparameter}. The problem of this technique arises when some hyperparameters are categorical or integer-valued which is the reason why extra approximations can lead to worse results and special treatment is needed \cite{garrido2020dealing}. Another possible technique for the surrogate model is using random forests \cite{hutter2011sequential}. All in all, this function estimates the machine learning model depending on the hyperparameter configuration and also the prediction uncertainty $ \sigma(\lambda) $. A second function called \textit{acquisition function} $ u(\lambda)$ is built based on the prediction distribution. This $ u $ is responsible for the trade-off between exploitation and exploration. This means that configurations that lead to better model performances are exploited and values where no much information is gathered are explored. There are many numerous different possibilities for this function  \cite{wilson2018maximizing} but the most used one is the \textit{expected improvement} (EI) which is calculated with

\begin{equation}
	E[I(\lambda)] = E[max(f_{min}-y), 0].
\end{equation}

If the model prediction y with configuration $ \lambda $ follows a normal distribution \cite{feurer2019hyperparameter}, it leads to 

\begin{equation}
	E[max(f_{min}-y), 0] = (f_{min}-\mu(\lambda)) \Phi(\frac{f_{min}-\mu(\lambda)}{\sigma}) + \sigma \phi (\frac{f_{min}-\mu(\lambda)}{\sigma})
\end{equation}

with $ \phi $ and $ \Phi $ being the standard normal density and standard normal distribution and $ f_{min} $ the best result so far. 


In each iteration, a new candidate configuration $ \lambda^+ $  is generated by optimizing the acquisition function $ u $. This $ u $ is much cheaper to evaluate than the $ f $ which includes learning of an expensive neural network which makes the optimization much easier.

The exact steps are presented in Algorithm \ref{alg:bayesian_opt} and Figure \ref{fig:bayesian_optimization} shows schematic iteration steps.

\begin{algorithm}[htbp!]
	\caption{Bayesian Optimization}\label{alg:bayesian_opt}
	\begin{algorithmic}
		\State Generate initial $\lambda^{(1)}, ..., \lambda^{(k)} $
		\State Initialize archive $A^{[0]} \gets ((\lambda^{(1)}, f(\lambda^{(1)})), ..., (\lambda^{(k)}, f(\lambda^{(k)})))$
		\State $ t \gets 1 $ 
		\While{Stopping criterion not met}
			\State Fit surrogate model $ (f(\lambda), \sigma(\lambda)) $ on $ A^{[t-1]} $
			\State Build acquisition function $ u(\lambda) $ from $ (\hat{f}(\lambda), \sigma(\lambda)) $
			\State Obtain proposal $ \lambda^{+} $ by optimizing $ u: \lambda^+ \in arg\max_{\lambda \in \Lambda} u(\lambda) $
			\State Evaluate $ f(\lambda^+)$
			\State Obtain $A^[t]$ by augmenting $ A^{[t-1]} $ with $ (\lambda^{(+)}, f(\lambda^{(+)})) $
			\State $ t \gets t+1 $
		\EndWhile
		\State return $ \lambda_{best} $: Best-performing $\lambda$ from archive or according to surrogates prediction
	\end{algorithmic}
\end{algorithm}

\begin{figure}[htbp!]
	\centering
	\includegraphics[scale=0.35]{figures/bayesian_optimization.png}
	\caption{ Schematic iteration steps of the bayesian optimization. The maximum of the acquisition function determines the next function evaluation (red dot in the middle). The goal is to find the minimum of the dashed line. The blue band is the uncertainty of the function. Taken from \cite{feurer2019hyperparameter}. }
	\label{fig:bayesian_optimization}
\end{figure}


First, $ k $ initial hyperparameter configurations are sampled and evaluated. This set is the starting archive $ A^{[0]} $. After that, the loop is executed as long as the stopping criterion is not met. This can be for example a budget, meaning a maximum number of function evaluations. The first step of the loop is to fit the surrogate model on the current archive. Then the acquisition function is made and optimized to get the next configuration $ \lambda^+ $. This point is evaluated and added to the archive. The overall result of the algorithm is the $ \lambda $ which is the hyperparameter configuration for the machine learning model with the overall best result.

\subsection{Other Techniques}
There are also other techniques for finding the best hyperparameters. Multi-fidelity optimization \cite{feurer2019hyperparameter} aims to probe the learning of model on a task with reduced complexity such as a subset of the data or less epochs for training the model for discovering the best configurations. For example, the learning curve can be predicted so that early stopping can be done if the prediction is not as good as the best model so far. There are also bandit-based selection methods that do not predict the learning curve but compare the different combinations on a small number of epochs and only performs the best ones. This can be done iteratively like it is done in \textit{successive halving} for hyperparameter optimization \cite{jamieson2016non}. The algorithm is very simple. It starts to evaluate all different combinations with very small budget. The best half of the candidates are then evaluated in the next iteration with double budget and so on until only one combination is left. In \cite{8030298}, a similar algorithm is presented. The authors use a model of the objective function (neural network depending on configurations) to find candidate hyperparameters. Those are then trained on a smaller number of epochs and the best ones then evaluated with higher budget. Also neural networks can be used for the optimization which was done by the authors in \cite{smithson2016neural}. Also, covariance matrix adaptation evolution strategy was implemented as an alternative to bayesian optimization in \cite{loshchilov2016cma}. 


\section{Sparse Grids}

Sparse grids are a useful tool to mitigate the \textit{curse of the dimensionality} by reducing the number of grid points. In the following, this technique is presented after the general numerical approximation of functions.

\subsection{Numerical Approximation of Functions}

\cite{b_splines}
Let $ f: \Omega \rightarrow R $ be a function defined on the unit interval $ \Omega = [0,1]^d $ in $ d $ dimensions. For simplicity, we first set $ d=1$. Now this function can be represented on a grid of level $ l \in \mathbb{N}_0 $ with $ 2^l + 1 $ grid points which are 
\begin{equation}
	x_{l,i} = i*h_l, \text{   } i = 0,...,2^l,
\end{equation}

with i being the index and $ h_l = 2^{-l} $ being the distance between the grid points. Each of them gets a basis function defined by 
\begin{equation}
	\varphi_{l,i}: [0,1] \rightarrow \mathbb{R}.
\end{equation}

There are different possibilities for the basis functions which will be presented later. For the simplicity, we present a simple example being the hat function defined by
\begin{equation}
	\varphi_{l,i}(x) = \max(1- |\frac{x}{h_l}-i|, 0).
\end{equation}
All in all, the space of functions that can be presented exactly by a linear combination is called the \textit{nodal space} $V_l$ with the assumption that the basis functions form a basis:
\begin{equation}
	V_l = \text{span}\{ \varphi_{l,i} | i = 0,...,2^l\}. 
\end{equation}

Every function $f: [0,1] \rightarrow \mathbb{R}$ can be interpolated by a the interpolant $ u $ defined by
\begin{equation}
	f_l = \sum_{i=0}^{2^l}\alpha_{l,i} \varphi_{l,i}, \forall i = 0,...,2^l: f_l(x_{l,i}) = f(x_{l,i})
\end{equation}

for constants $ \alpha_{l,i} \in \mathbb{R} $. An example can be seen in Figure \ref{fig:interpolant}.

\begin{figure}[htbp!]
	\centering
	\includegraphics[scale=0.5]{figures/weighted_sum.png}
	\caption{ Interpolation of the function $ f $ (black line) by its interpolant $ u $ (red, dashed) in the nodal basis. Level of the grid is 3 and hat functions are used. Taken from \cite{pfluger2010spatially}. }
	\label{fig:interpolant}
\end{figure}

On the left side, the function f (black line) can be seen with a grid of level 3. On the right side, the interpolant u as a linear combination of the basis functions (hat functions centered on the grid points) can be seen. This approach is the nodal basis. The second possibility is called hierarchical basis and the index set is $I_l^h = \{i \in \mathbb{N} | 1 \le i \le i \le s^l-1, i \text{odd}\}$. The hierarchical subspaces are then 
\begin{equation}
	W_l = \text{span}\{ \varphi_{l,i}(x) | i \in I_l^h\}.
\end{equation}

The same nodal space $ V_l $ can be obtained with the hierarchical subspaces with 
\begin{equation}
	V_l = \bigoplus_{i \le l} W_i.
\end{equation}

An example can be seen in Figure \ref{fig:hierarchical_basis}.
\begin{figure}[htbp!]
	\centering
	\includegraphics[scale=0.5]{figures/hierarchical_basis.png}
	\caption{ Hierarchical subspaces up to level 3 on the left. On the right, nodal spaces up to level 3. The combination of $ W_1 $ up to $ W_3 $ is the same space as $ V_3 $. Taken from \cite{pfluger2010spatially}. }
	\label{fig:hierarchical_basis}
\end{figure}

On the left, you can see the hierarchical subspaces up to level 3. All in all, combined they span the same space as $ V_3 $. In the hierarchical case, a function $ f $ can also be interpolated by its interpolant $ u $ by 
\begin{equation}
	u = \sum_{i \in I_l^h}\alpha_{l,i} \varphi_{l,i}, \forall i = 0,...,2^l: u(x_{l,i}) = f(x_{l,i}).
\end{equation}

An example can be seen in Figure \ref{fig:weighted_sum_hierarchical}.

\begin{figure}[htbp!]
	\centering
	\includegraphics[scale=0.5]{figures/weighted_sum_hierarchical.png}
	\caption{ Interpolation of the function $ f $ (black line) by its interpolant $ u $ (red, dashed) in the hierarchical basis. Level of the grid is 3 and hat functions are used. Taken from \cite{pfluger2010spatially}.}
	\label{fig:weighted_sum_hierarchical}
\end{figure}

To get into higher dimensions $ d > 1 $, we use the tensor product. The domain is now $ \Omega = [0,1]^d $ and the level is defined by the level per dimension meaning $ \vec{l} = (l_1, ..., l_d) \in \mathbb{N}_0^d $. The index set is then
\begin{equation}
	I_{\vec{l}} = \{ \vec{i} | 1 \le i_j \le 2^{l_j} -1 , i_j \text{odd}, 1 \le j \le d \}
\end{equation}

and the subspaces 

\begin{equation}
	W_{\vec{l}} = \text{span}\{ \varphi_{\vec{l},\vec{i}}( \vec{x} ) | \vec{i} \in I_{\vec{l}}\}
\end{equation}

with the basis functions $ \varphi_{\vec{l},\vec{i}} = \prod_{j=1}^{d} \varphi_{l_j,i_j}(x_j) $ which are constructed with the tensor product. The function space $ V_n $ is constructed by
\begin{equation}
	V_n = \bigoplus_{|\vec{l}|_\infty \le n} W_l
\end{equation}
with $ |\vec{l}| = \max_{1 \le i \le d} |d_i| $. Again, a function can be interpolated by its interpolant $ u $ with
\begin{equation}
	u = \sum_{|\vec{l}|_\infty \le n, \vec{i} \in I_{\vec{l}}}\alpha_{\vec{l},\vec{i}} \varphi_{\vec{l},\vec{i}}, \forall \vec{i} \in I_{\vec{l}}: u(x_{\vec{l},\vec{i}}) = f(x_{\vec{l},\vec{i}}).
\end{equation}

The resulting regular grid has then $ (2^n - 1)^d $ basis points. An example of a basis function in two dimensions can be seen in Figure \ref{fig:2d_basis}. It is constructed by the tensor product of two 1d hat functions. 

\begin{figure}[htbp!]
	\centering
	\includegraphics[scale=0.26]{figures/2d_basis.png}
	\caption{ Example of a basis function in two dimensions. It is constructed with the tensor product of two 1d hat functions. Taken from \cite{garcke2013sparse}. }
	\label{fig:2d_basis}
\end{figure}

In the higher dimensional case, the grid can also be constructed hierarchically. The proof that the hierarchical splitting given by 
\begin{equation}
	V_{\vec{l}} = \bigoplus_{\vec{m} = 0}^{\vec{l}} W_{\vec{m}}
\end{equation}
with $ W_{\vec{l}} = \text{span}\{\varphi_{\vec{l}, \vec{i}} | \vec{i} \in I_{\vec{l}}\}, I_{\vec{l}} = I_{l_1} \times ... \times I_{l_d} $ holds for the basis with hat functions can be found in \cite{b_splines}.

\subsection{Adaptive Sparse Grids}
The problem of regular grids is the \textit{curse of the dimensionality} because of the high number of grid points in higher dimensions. This is tackled by sparse grids \cite{zenger1991sparse, bungartz2004sparse} by reducing this number. The first technique to achieve this is by just leaving out subspaces. The resulting sparse function space is given by 
\begin{equation}
	V_{n}^1 = \bigoplus_{|\vec{l}|_1 \le n+d-1 } W_{\vec{l}} \subset V_n.
\end{equation}
An example with $ n = 3 $ can be seen in Figure \ref{fig:sparse_grid}.
\begin{figure}[htbp!]
	\centering
	\includegraphics[scale=0.5]{figures/subspaces_twodims.png}
	\caption{ Two dimensional example of a sparse grid with $ n = 3 $. Left, the subspaces $ W_{\vec{l}} $ can be seen and on the right is the resulting sparse grid. Taken from \cite{garcke2013sparse}. }
	\label{fig:sparse_grid}
\end{figure}

An interpolant $ u_n $ of a function $ f $ is then constructed by
\begin{equation}
	u_l = \sum_{ |\vec{l}|_1 \le l+d-1 } \sum_{ \vec{i} \in I_{\vec{l}} } \varphi_{\vec{l}, \vec{i}} \alpha_{\vec{l},\vec{i}}
\end{equation}
where the $ \alpha_{\vec{l},\vec{i}} $ are the coefficients of the basis functions \cite{obersteiner2022spatially}.

A second approach for sparse grids exists. The so-called \textit{combination technique} \cite{griebel1990combination} combines anisotropic full grids to get the same subspace as the conventional sparse grid approach. This has the advantage that we can use normal full grid operations on each subspace which will then be combined. This implies the possibility of parallelization. The combined solution can be computed with 
\begin{equation}
	u_l^c = \sum_{ \vec{l} \in I } u_{\vec{l}} c_{\vec{l}}
\end{equation}
where $ \vec{l} $ is the level vector of the full grid solution $ u_{\vec{l}} $, $ c_{\vec{l}} $ is a scalar factor, and $ I $ is the set of included level vectors. For a standard sparse grid, this evaluates to 
\begin{equation}
	u_l^c = \sum_{ q = 0 }^{d-1} (-1)^q \binom{d-1}{q} \sum_{\vec{l} \in I_{l,q} } u_{\vec{l}}
\end{equation}
with $ I_{l,q} = \{ \vec{l} \in \mathbb{N}_0^d | ||\vec{l}||_1 = l+d-1-q \} $ \cite{obersteiner2021generalized}. An example of the 2-dimensional combination technique can be seen on the left side of Figure \ref{fig:combi_technique}.

With the normal combination technique, this grid is still symmetric and focuses on a low global error. Especially in optimization or data driven problems where the points are not distributed equally in the domain, special regions are of interest. In the case of optimization which is our focus, the errors around the extrema have to be interpolated more exactly than other regions. This is the reason why we use \textit{refinement}. In the case of dimension-adaptive refinement \cite{hegland2002adaptive}, more grid points are added in the dimensions of higher relevance. 

\begin{figure}[htbp!]
	\centering
	\includegraphics[scale=0.18]{figures/normal_combi_technique.png}
	\includegraphics[scale=0.18]{figures/adaptive_combination_technique.png}
	\caption{ Example of the 2-dimensional combination technique. Here the blue regular grids are added and the red ones are subtracted. On the left, the normal combination technique can be seen and on the right is an dimension-adaptive version. Taken from \cite{pfluger2010spatially}. }
	\label{fig:combi_technique}
\end{figure}


In contrast to the previously mentioned refinement concentrating on whole dimensions, the \textit{spatially adaptive refinement} directly adds grid points where the discretization error is still high. An example of the spatially adaptive combination technique presented by \cite{obersteiner2021generalized} can be seen in figure \ref{fig:spatially_adaptive_combi_technique}. In this example, the basis points of the component grids are no longer equidistant because refinement was already made. 
\\

All in all, Table \ref{tab:comparison_grids} shows the comparison of full grids, sparse grids, and the combination technique in terms of number of points and interpolation accuracy \cite{pfluger2010spatially}.


\begin{table}[h!]
	\centering
	\begin{tabular}{|c c c|} 
		\hline
		Grid & Number grid points & Accuracy \\
		\hline
		Full grid & $ \mathcal{O}(h_n^2) $ & $ \mathcal{O}(2^{nd}) $ \\
		Sparse grid & $ \mathcal{O}(h_n^{-1}(\text{log } h_n^{-1})^{d-1}) $ & $ \mathcal{O}(h_n^{2}(\text{log } h_n^{-1})^{d-1}) $ \\
		Combination technique & $ \mathcal{O}(d(\text{log } h_n^{-1})^{d-1}) \times  \mathcal{O}(h_n^{-1}) $ & $ \mathcal{O}(h_n^{2}(\text{log } h_n^{-1})^{d-1}) $\\
		\hline
	\end{tabular}
	\caption{ Comparison of sparse grids, full grids, and the combination technique in terms of number of grid points and the accuracy.}
	\label{tab:comparison_grids}
\end{table}

\begin{figure}[htbp!]
	\centering
	\includegraphics[scale=0.2]{figures/spatially_adaptive_combi_technique.png}
	\caption{ Example of the spatially adaptive combination technique in two dimensions. Taken from \cite{obersteiner2021generalized}. }
	\label{fig:spatially_adaptive_combi_technique}
\end{figure}

\subsection{Basis Functions for Sparse Grids}

So far, we only considered the simple case of the hat function on the support points. Besides them, there are other possibilities, for example piecewise d-polynomial, wavelet, and B-spline basis functions. For the first two cases, refer to \cite{pfluger2010spatially, bungartz1998finite, bungartz2004sparse} for further readings. In this thesis, we will concentrate on the B-spline basis for the sparse grids as the hat function is not continuously differentiable \cite{b_splines}. This is the reason why we can not compute globally continuous gradients which is a problem for the optimization. The general cardinal B-spline with degree $ p \in \mathbb{N}_0 $ is defined by 
\begin{equation}
	b^p(x) = \begin{cases} 
				\int_0^1 b^{p-1}(x-y) \text{d}y				& p \geq 1 \\
				\chi_{[0,1[}(x) 							& p=0 \\
			\end{cases}
\end{equation}
with $ \chi_{[0,1[} $ being the characteristic function of the half-open unit interval \cite{hollig2013approximation}. The $ b^p $ as defined above has the following 8 properties:

\begin{enumerate}
	\item compactly supported on $ [0, p+1] $
	\item symmetric and $ 0 \le b^p \le 1 $
	\item weighted combination of $ b^{p-1} $ and $ -b^{p-1} $
	\item piecewise polynomial of degree $ p $
	\item $ \frac{d}{dx} b^p $ is the difference of $ b^{p-1} $ and $ -b^{p-1} $
	\item has unit integral 
	\item is the convolution of $ b^{p-1} $ and $ b^{0} $
	\item hat function and gaussian function are special cases
\end{enumerate}

This is the case for uniform B-splines. For adaptive grids, the distances between basis points are not always uniform. This is the reason why we need also non-uniform B-splines. Let $ m, p \in \mathbb{N}_0 $ and $ \xi = (\xi_0, ... , \xi_{m+p}) $ be an increasing sequence of real numbers called \textit{knot sequence}. For $ k=0,..., m-1 $, the non-uniform B-spline is defined by 
\begin{equation}
	b^p_{k,\xi}(x) = \begin{cases} 
		\frac{x-\xi_k}{\xi_{k+p}-\xi_k}b^{p-1}_{k,\xi}(x) + \frac{\xi_{k+p+1}-x}{\xi_{k+p+1}-\xi_{k+1}}b^{p-1}_{k+1,\xi}(x)				& p \geq 1 \\
		\chi_{[\xi_{k},\xi_{k+1}[}(x) 							& p=0 \\
	\end{cases}
\end{equation}

This definition and the proof that the hierarchical splitting also holds for using the B-splines for restricted functions can be found in \cite{b_splines}.
