% !TeX root = ../main.tex
% Add the above to each chapter to make compiling the PDF easier in somexcv editors.

\chapter{Related Work}\label{chapter:theoretical_background}

\section{Hyperparameter Optimization}

Most machine learning models have parameters that have to be defined before the learning phase. They are called hyperparameters and strongly influence the behavior of the model. One example is the number of epochs of the learning phase of a neural network. There are different techniques for the optimization of hyperparameters and they all define the machine learning model as a black box function $ f $ with the hyperparameters as input and the resulting performance as output. The overall goal is to find a configuration $ \lambda_{min} $ from $ \Lambda = \Lambda_1 \times \Lambda_2 \times ... \times \Lambda_N $ that minimizes the function $ f $ with $ N $ hyperparameters with 
\begin{equation}
	\label{eq:optimization}
	\lambda_{min} = \text{arg} \min_{\lambda \in \Lambda} f(\lambda) .
\end{equation}

In our case, the function f is a machine learning algorithm that is trained on a training set and evaluated on a testing set. With this, the minimization of e.g. the loss of the model optimizes the decisions it is making which leads to better prediction results. Note that one function evaluation of f is usually very expensive as the training of a machine learning model with many parameters and weights takes much time. The data set consists of $ \{ (x_i, y_i) | x_i \in X, y_i \in Y, 0 \le i \le m \} $ with m being the number of data samples. The $ x_i $ is the input data to the model and the goal is that 
\begin{equation}
	\forall i: M(x_i) = y_i.
\end{equation}
where M is the model. In the context of supervised learning, the whole data set is split into a training set which is used to optimize the model and a testing set to evaluate the performance on new, unseen data \cite{supervised_learning}.

All in all, the goal is get evaluation scores on the testing data set which can be achieved with Equation \ref{eq:optimization}. 
\cite{feurer2019hyperparameter,bischl2021hyperparameter}

In the following, different techniques for the optimization are presented and discussed with their advantages and disadvantages.

\subsection{Grid Search}
The idea of the first approach for the optimization is to discretize the domains of each hyperparameter and evaluate each combination. This suffers from the curse of the dimensionality as it scales exponentially with the number of hyperparameters. For $ d $ parameters and $ n $ values per hyperparameter, $ n^d $ different configurations are possible which all have to be evaluated. 

One advantage of this method is that it is easy to implement and very simple. Also, the whole search space is explored evenly.

On the other hand, the curse of the dimensionality makes it very slow if the function evaluations are very expensive. For most machine learning 




\subsection{Random Search}

\cite{random_search}

\subsection{Bayesian Optimization}

\subsection{Other Techniques}


\section{Sparse Grids}

\subsection{Numerical Approximation of Functions}

\subsection{Adaptive Sparse Grids}
